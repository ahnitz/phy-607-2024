%% Welcome to the syllabus! This file isn't really created as an example, but I thought it might be interesting for you to see how I wrote it

\documentclass[12pt]{article}
\usepackage[utf8]{inputenc}

\usepackage[T1]{fontenc}

\usepackage{mathptmx}
\usepackage{helvet}
\usepackage{array}
\renewcommand\familydefault{\sfdefault}

\usepackage{graphicx}
\usepackage{amsmath}
\usepackage{amssymb}
\usepackage{physics}
\usepackage[colorlinks,urlcolor=red]{hyperref}
\usepackage{cleveref}
\usepackage{subcaption}
\usepackage{booktabs}

%can uncomment to try it
%it seems to work for me, but the package is deprecated according to the docs. hopefully something new comes along!
\usepackage[tagged,flatstructure]{accessibility}

\makeatletter
\@beginparpenalty=10000
\makeatother

\title{PHY 607 - Computational physics \\ Fall 2024 syllabus}
\author{Prof. Alexander Nitz -- \href{mailto:ahnitz@syr.edu}{ahnitz@syr.edu}}
\date{}

\begin{document}

\maketitle

\section*{Class information}

Welcome to PHY 607! This three credit course is an introduction to using computers to solve problems relevant for physics research. The goal is to provide a foundation in programming, best practices, algorithms, and collaboration upon which one can build to employ more sophisticated techniques. The audience includes both students new to programming as well as those with some previous experience. The course is composed of
lectures and tutorials on best practices and projects to build experience.

\subsection*{Instructor}

\begin{minipage}{0.5\textwidth}
  Prof. Alexander Nitz \\
  Office hours: Monday 3-5 pm (or by appointment)
\end{minipage}
\begin{minipage}{0.5\textwidth}
  \begin{flushright}
  email: \href{mailto:ahnitz@syr.edu}{ahnitz@syr.edu} \\
  office: Physics 263-I
  \end{flushright}
\end{minipage}

\subsection*{Learning Objectives}
Students will be able to
\begin{itemize}
\item create basic programs implementing numerical algorithms in python,
\item learn best practices for validating and testing their programs,
\item compare their results with more sophisticated techniques available in external packages,
\item work collaboratively using version control,
\item document results.
\end{itemize}

\subsection*{Meeting times}

The scheduled class times are Tuesday and Thursday 12:30 - 13:50 PM in room 208. These times will be used as default times to schedule a number of different class types:
\begin{description}
\item[Lectures and Discussions] When introducing the course, lectures on practices and concepts, and new projects.
\item[Small group sessions] When collaborating on a project, I will schedule discussion sessions with individual groups to review progress and issues.
\item[Open lab sessions] For all other class times, plan to be present, working on projects. This gives you quick access to the professor and other students to answer questions.

\end{description}
The schedule details for individual days will be determined on an as-needed basis.

\subsection*{Books and materials}

\begin{itemize}
\item A MacOS/Linux compatible computer capable of using python and the SciPy programming stack is required for this course.
\item No textbook is required for this course. Suggested online resources will be supplied as needed.
\end{itemize}

\section*{Grading}

\begin{table}[h]
  \caption{Grade breakdown}
\begin{center}
  \begin{tabular}{cr}
	\toprule
    Four major projects & scored at 5, 15, 25 and 25\%, respectively \\
    assignments & 20\% \\
    Collaboration & 10\% \\
    \bottomrule
  \end{tabular}
\end{center}
\end{table}

Throughout the term, assignments will be used for class and project organization. The first major project will be completed individually, but for the rest you will work in assigned groups.

\begin{table}[h]
  \caption{Letter grades}
\begin{center}
  \begin{tabular}{cl}
	\toprule
	Score & Grade \\
	\midrule
	90+     &         A \\
   85-89    &         A-\\
   80-84    &         B+\\
   75-79    &         B	\\
   70-74    &         B-\\
   65-69    &         C+\\
   60-64    &         C	\\
   55-59    &         C-\\
   0-54    &         F \\
   \bottomrule
  \end{tabular}
\end{center}
\end{table}
\newpage

\section*{Course calendar}
The following key dates are tentatively set for the semester, and may be adjusted based on progress in the course. Unlisted class dates will be scheduled as described in the section on meeting times, above. Timelines and topics are subject to adjustment, but may include topics related to optimization, ODEs, PDEs, monte-carlo methods, and markov chains.

\begin{table}[h]
  \caption{Course Schedule}
\begin{center}
  \begin{tabular}{|p{3cm}|m{10cm} | @{}m{0pt}@{}}
     \toprule
     \textbf{Week} & \textbf{Topic} \\\hline
    Week 1 & Introduction, class requirements, pseudocode, bash/terminal \\ \hline
    Week 2 & Intro do python programming, version control \\
    Week 3 & \textbf{code practices}: plotting / io, validation, numerical error \textbf{physics}: integration/odes \\\hline
    Week 4 & \textbf{code practices}: classes, style, sandboxes \textbf{physics}: integration/runge kutta \\\hline
    Week 5 & \textbf{code practices}: profiling, computational complexity \textbf{physics}: random numbers \\\hline
    Week 6+7 & \textbf{code practices}: programs, numpy \textbf{physics}: monte-carlo methods, rejection sampling, monte-carlo integration \\\hline
    Week 8+9 & \textbf{code practices}: packages \textbf{physics}: optimization methods, fitting \\\hline
    Week 10+11 & \textbf{code practices}: identifying errors / debugging \textbf{physics}: likelihoods, bayes theorem, markov chains \\\hline
    Week 12 + 13 & \textbf{code practices}: \textbf{physics}: nbody / finite element \\\hline
    Week 13+14 & Specials topics and Project 4 tutorials \\
    \bottomrule
  \end{tabular}
\end{center}
\end{table}

\begin{table}[h]
  \caption{Key Dates}
\begin{center}
  \begin{tabular}{l|r}
    \toprule
    2023/8/27 & First class \\
    2023/8/29 & Project 1 Intro \\
    2023/9/27 & Project 1 due \\
    2023/10/03 & Project 2 intro \\
    2023/10/22 & Project 2 due \\
    2023/10/22 & Project 3 intro \\
    2023/11/17 & Project 3 due \\
    2023/11/14 & Project 4 intro \\
    2023/12/10 & Last day of class and project 4 due \\
    \bottomrule
  \end{tabular}
\end{center}
\end{table}

\clearpage

\section*{Course policy on missed work}

All course projects are required. In the event of unforeseen circumstances causing difficulty completing work on-time, you must communicate in advance. Accommodations will be made on a case-by-case basis.

\section*{Course policy on academic integrity}

You are encouraged to discuss everything in the course with your peers and to search online for example solutions to programming questions. But all code and documentation you submit must be your own work.

\section*{Use of Artifical Intelligence}

Based on our specified learning outcomes, the partial use of artificial intelligence as a tool, with disclosure and citation, is permitted in this course for the purposes of validation, investigation, and debugging. Students do not need to ask permission to use these tools before starting an assignment, but they must explicitly and fully indicate which tools were used and describe how they were used. All information, results, or inferences obtained from AI tools must also be vetted by the student.

\section*{Online etiquette}

Classes are intended to be in person and a hybrid experience will not be offered. In the event that it is necessary
to hold a class meeting or discussion online, there are a few simple things you can do that will make life easier for everyone and keep the environment friendly:
\begin{itemize}
\item First, make sure you are aware of what you are transmitting. (If you’re sending video over the internet, for instance, make sure you’re wearing appropriate clothing!) Treat online class meetings as you would treat classes in person.
\item If you are in a larger group, keep your microphone muted unless you are actively talking; this is especially important if you are in a noisy place or if you are using a built-in microphone on a laptop or cellphone.
\item If you are using a platform such as Zoom that allows you to change your name, please modify your name to include your personal pronouns, e.g. “Alex Nitz (he/him)”. If you are referring to someone whose pronouns you do not know, it is always appropriate to use they/them.
\end{itemize}

\section*{Inclusion}

Everyone in this class is an equally-valued member of this university and our community. We expect you to treat your classmates as honored colleagues in the collective endeavor we are all involved in: to understand the natural world and use that understanding to improve our society.

In particular, bias against or denigration of anyone in our class because of their gender or how they express it, their sexual orientation, their religion, their national origin, their race or ethnicity, or a disability they may have will not be tolerated. If you are the target of this sort of bias or if you witness it, please report it directly to me and I will take swift action. If you don’t feel comfortable talking to me, you may report it anonymously to the Physics Department at
\href{https://syracuseuniversity.qualtrics.com/jfe/form/SV_9pORpTKnq6pLeyF}{the department confidential complaint form}.

\newpage

\section*{Syracuse University Policies}
Students should review the University’s policies regarding: Diversity and Disability: \url{https://www.syracuse.edu/life/accessibilitydiversity/};  the Religious Observances Notification and Policy: \url{http://supolicies.syr.edu/studs/religious_observance.htm}; and Orange SUccess: \url{http://orangesuccess.syr.edu/getting-started-2/}.

\subsection*{University Attendance Policy}
Attendance in classes is expected in all courses at Syracuse University. Students are expected to arrive on campus in time to attend the first meeting of all classes for which they are registered. Students who do not attend classes starting with the first scheduled meeting may be academically withdrawn as not making progress toward degree by failure to attend. Instructors set course-specific policies for absences from scheduled class meetings in their syllabi.

It is a federal requirement that students who do not attend or cease to attend a class to be reported at the time of determination by the faculty. Faculty should use “ESPR” and “MSPR” in Orange Success to alert the Office of the Registrar and the Office of Financial Aid. A grade of NA is posted to any student for whom the Never Attended flag is raised in Orange SUccess. More information regarding Orange SUccess can be found here, at \url{http://orangesuccess.syr.edu/getting-started-2/}.

\subsection*{Disability-Related Accommodations}

Syracuse University values diversity and inclusion; we are committed to a climate of mutual respect and full participation. There may be aspects of the instruction or design of this course that result in barriers to your inclusion and full participation in this course. I invite any student to contact me to discuss strategies and/or accommodations (academic adjustments) that may be essential to your success and to collaborate with the Center for Disability Resources (CDR) in this process.

If you would like to discuss disability-accommodations or register with CDR, please visit Center for Disability Resources. Please call (315) 443-4498 or email disabilityresources@syr.edu for more detailed information.

The CDR is responsible for coordinating disability-related academic accommodations and will work with the student to develop an access plan. Since academic accommodations may require early planning and generally are not provided retroactively, please contact CDR as soon as possible to begin this process.

Furniture may be placed in your classrooms for specific use by students with disabilities. This furniture is generally labelled with signs that request that it is not to be moved per the Center for Disability Resources. Students with disabilities rely on this furniture to remain where it is placed within the classrooms.

Note that some students may be accompanied to classes by a service animal. Students are not required to provide prior notice of this to the University. If it is not readily apparent that the student is disabled, students can be asked two questions, do you have this animal due to a disability and what tasks does it perform for you? Emotional Support Animals are different than services animals and are typically only approved to reside in a student’s housing location/room.

\subsection*{Academic Integrity Policy}

As a pre-eminent and inclusive student-focused research institution, Syracuse University considers academic integrity at the forefront of learning, serving as a core value and guiding pillar of education. Syracuse University’s Academic Integrity Policy provides students with the necessary guidelines to complete academic work with integrity throughout their studies. Students are required to uphold both course-specific and university-wide academic integrity expectations such as crediting your sources, doing your own work, communicating honestly, and supporting academic integrity. The full Syracuse University Academic Integrity Policy can be found by visiting class.syr/edu, selecting, “Academic Integrity,” and “Expectations and Policy.”

Upholding Academic Integrity includes the protection of faculty’s intellectual property. Students should not upload, distribute, or share instructors’ course materials, including presentations, assignments, exams, or other evaluative materials without permission. Using websites that charge fees or require uploading of course material (e.g., Chegg, Course Hero) to obtain exam solutions or assignments completed by others, which are then presented as your own violates academic integrity expectations in this course and may be classified as a Level 3 violation. All academic integrity expectations that apply to in-person assignments, quizzes, and exams also apply online.

Students found in violation of the policy are subject to grade sanctions determined by the course instructor and non-grade sanctions determined by the School or College where the course is offered. Students may not drop or withdraw from courses in which they face a suspected violation. Any established violation in this course may result in course failure regardless of violation level.

\subsection*{Discrimination or Harassment}

The University does not discriminate and prohibits harassment or discrimination related to any protected category including creed, ethnicity, citizenship, sexual orientation, national origin, sex, gender, pregnancy, disability, marital status, age, race, color, veteran status, military status, religion, sexual orientation, domestic violence status, genetic information, gender identity, gender expression or perceived gender.

Any complaint of discrimination or harassment related to any of these protected bases should be reported to Sheila Johnson-Willis, the University’s Chief Equal Opportunity \& Title IX Officer. She is responsible for coordinating compliance efforts under various laws including Titles VI, VII, IX and Section 504 of the Rehabilitation Act. She can be contacted at Equal Opportunity, Inclusion, and Resolution Services, 005 Steele Hall, Syracuse University, Syracuse, NY 13244-1120; by email: titleix@syr.edu; or by telephone: 315-443-0211.

Faculty who wish to include general language about discrimination or harassment in their syllabi may include the two preceding paragraphs. Faculty who wish to include language specific to sexual violence or harassment may include the following statement on their syllabi:

Federal and state law, and University policy prohibit discrimination and harassment based on sex or gender (including sexual harassment, sexual assault, domestic/dating violence, stalking, sexual exploitation, and retaliation). If a student has been harassed or assaulted, they can obtain confidential counseling support, 24-hours a day, 7 days a week, from the Sexual and Relationship Violence Response Team at the Counseling Center (315-443-8000, Barnes Center at The Arch, 150 Sims Drive, Syracuse, New York 13244). Incidents of sexual violence or harassment can be reported non-confidentially to the University’s Title IX Officer (Sheila Johnson Willis, 315-443-0211, titleix@syr.edu, 005 Steele Hall). Reports to law enforcement can be made to the University’s Department of Public Safety (315-443-2224, 005 Sims Hall), the Syracuse Police Department (511 South State Street, Syracuse, New York, 911 in case of emergency or 315-435-3016 to speak with the Abused Persons Unit), or the State Police (844-845-7269). I will seek to keep information you share with me private to the greatest extent possible, but as a professor I have mandatory reporting responsibilities to share information regarding sexual misconduct, harassment, and crimes I learn about with the University’s Title IX Officer to help make our campus a safer place for all.

\end{document}
